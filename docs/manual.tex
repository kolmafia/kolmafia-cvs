\documentclass[twocolumn,letterpaper]{report}
\setlength{\textheight}{8.875in}
\setlength{\textwidth}{6.875in}
\setlength{\columnsep}{0.3125in}
\setlength{\topmargin}{0in}
\setlength{\headheight}{0in}
\setlength{\headsep}{0in}
\usepackage{amsmath,amsthm}
\usepackage{graphicx}
\newtheorem{tip}{Tip}[section]

\begin{document}
\title{Discovering KoLmafia\\A KoLmafia End-User Manual}
\author{KoLmafia Development Team\\http://kolmafia.sourceforge.net/}
\maketitle

\tableofcontents

\chapter{I'm Here}

\section{Overview}

You can think of this manual as an origami book; instead of showing you all the aspects of folding paper, you will instead see some end results and then receive descriptions on how these end results can be achieved.  In doing so, this document intends to help you familiarize yourself with KoLmafia so that you'll be able to find some of its other hidden features on your own and perhaps even make suggestions on how the project can be improved.

As some of the end results are being described, it's possible to run into features which are not immediately apparent from the KoLmafia user interface.  Then, just like every good tourist guidebook (see, we move from an origami analogy to a tourism analogy), the manual will detour for a moment and briefly introduce the hidden feature (or "gem" as many guidebooks call them).

\begin{tip}
When that happens, you will be notified by one of these little italicized sections as the hidden gem is being discussed.  So, if you were just looking through the manual to see if there were any features you might not know about, be sure to read these sections.
\end{tip}

In other words, this is not a technical manual; it's an art book (another analogy transition).  And like you would expect in an art book, there will be pictures.  However, because the theme used by OS X ("Aqua") is different from the themes used by other operating systems ("Ocean"), the screenshots you see in this manual will be subtly different from what you see while you are personally running KoLmafia.

Before you start reading any more this manual, it is suggested that you visit the KoLmafia home page hosted on Sourceforge and read through the non-technical documentation (ie: everything that's not the manual). By doing so, you'll get a better understanding of KoLmafia's over-arching goals, which will greatly improve your experience in reading this manual.

This manual is written in a typesetting language called LaTeX.  In case you have no idea what LaTeX is (or rather, you suspect that you know what it is, but the word does not make you think of a mathematical text book), a friendly search engine should tell you everything.  Like most LaTeX documents, the ritual sacrifice of a tree which seems to be required in order to read this it properly is completely intentional.

Finally, the chapter titles which you encountered on the table of contents are song titles for tracks which can be found an album produced by Kajiura Yuki and Nanri Yuuka titled, "Destination".  This album conveniently had eleven tracks (that's ridiculous, it's not even funny) and was a great source of inspiration for the less-muted sections of the manual.

\section{Getting Started}

\chapter{Destination}
\chapter{Nowhere}
\chapter{Akatsuki no Kuruma}
\chapter{Daremo Inai Basho}
\chapter{Seiya}
\chapter{Shizukana Kotoba}
\chapter{Futari}
\chapter{Hitomi no Kakera}
\chapter{Nostalgia}
\chapter{Inside Your Heart}
\end{document}