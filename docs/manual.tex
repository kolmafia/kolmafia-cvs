\documentclass[twocolumn,letterpaper]{report}
\setlength{\textheight}{8.875in}
\setlength{\textwidth}{6.875in}
\setlength{\columnsep}{0.3125in}
\setlength{\topmargin}{0in}
\setlength{\headheight}{0in}
\setlength{\headsep}{0in}
\usepackage{amsmath,amsthm}
\usepackage{graphicx}
\newtheorem{tip}{Tip}[section]

\begin{document}
\title{Discovering KoLmafia\\A KoLmafia End-User Manual}
\author{KoLmafia Development Team\\http://kolmafia.sourceforge.net/}
\maketitle

\tableofcontents

\chapter{I'm Here}

\section{Overview}

Welcome to the KoLmafia end-user manual!  This document is intended to help you, the end user, familiarize yourself with a small Java application called KoLmafia by introducing you to the motivations and design decisions behind the KoLmafia feature set.

If you visit the KoLmafia home page, you will find a description of the beginnings of the KoLmafia project as well as a summary of its over-arching goals, ideals and mission statements.  We invite you to read over these documents before using KoLmafia so you have a clear picture of what this project aims to achieve.

This manual is written in a typesetting language called LaTeX.  In case you have no idea what LaTeX is (or rather, you suspect that you know what it is, but the word does not make you think of a mathematical text book), a friendly search engine should give you a straightforward description.

The important thing to note from that piece of information is that, like most LaTeX documents, this manual is intended to be printed, not navigated while online.  Therefore, the ritual sacrifice of a tree which seems to be required in order to read this document properly is completely intentional.

Finally, the chapter titles which you encountered on the table of contents are song titles for tracks which can be found an album produced by Kajiura Yuki and Nanri Yuuka titled, "Destination".  This album conveniently had eleven tracks (that's ridiculous, it's not even funny) and was a great source of inspiration for the less-muted sections of the manual.

\section{Introduction}

KoLmafia is an application developed by a group of people who play the game Kingdom of Loathing, just like you.  Therefore, there are a lot of things that were added in order to make things more convenient for us, and it's entirely possible that someone else (like you) will find these additions useful.  This manual will focus on introducing you to these things.

Based on this, what one might reasonably expect is that the manual would go over what can be done on each screen of KoLmafia (there are several dozen such screens), documenting them in detail.  Please understand that the manual will do nothing of the sort, because this involves a tedious writing process and involves a monotonous reading process as well.

Instead, this manual is intended to help you discover KoLmafia, and therefore the idea behind each of the chapters and sections of this manual is to specify an objective and then demonstrate how this can be achieved using KoLmafia.  You can think of it like an origami book; the book does not show you everything which can be done by folding paper, but it shows you some end results and then demonstrates how these end results can be achieved.

Along the way, though, it's possible to run into features which are not immediately apparent from the KoLmafia user interface.  Then, just like every good tourist guidebook (see, we move from an origami analogy to a tourism analogy), the manual will detour for a moment and briefly introduce the hidden feature (or "gem" as many guidebooks call them).

\begin{tip}
When that happens, you will be notified by one of these little italicized sections as the hidden gem is being discussed.  So, if you were just looking through the manual to see if there were any features you might not know about, be sure to read these sections.
\end{tip}

Finally, KoLmafia is currently being developed on a Macintosh running OS X; as a result, the screenshots you will see scattered throughout the manual will not match the screens you would find on Windows or Linux.  The differences exist because the theme used by OS X ("Aqua") is different from the themes used by other operating systems ("Ocean"), and OS X uses a single global menu bar for all of its applications, so KoLmafia's windows will appear to not have any menus.

\section{Getting Started}

\chapter{Destination}
\chapter{Nowhere}
\chapter{Akatsuki no Kuruma}
\chapter{Daremo Inai Basho}
\chapter{Seiya}
\chapter{Shizukana Kotoba}
\chapter{Futari}
\chapter{Hitomi no Kakera}
\chapter{Nostalgia}
\chapter{Inside Your Heart}
\end{document}